\documentclass[10pt,draftclsnofoot,onecolumn]{article}

\usepackage[letterpaper, total={7in, 9.5in}]{geometry}
\usepackage[utf8]{inputenc}
\usepackage{times} 
\usepackage[T1]{fontenc}         
\usepackage{pslatex}
\usepackage{bibentry}

\pagestyle{empty}

\bibliographystyle{plain}

\title{GFR Team 1 Problem Statement}

\author{Nicholas Milford}

\date{\today}

\begin{document}
\maketitle

\begin{abstract}
\noindent
For Oregon State University's Global Formula Racing team to continue their main goal of winning competitions, it is necessary to include a competitive submission into the new driverless racing category. For a submission to be competitive, the driverless race car must be able to perform in several events each with unique conditions that require different optimizations. A working prototype of a submission will require implementing software solutions to observe the environment of the race track, plan a path through the track, and execute that plan through the car. This project is the design and development of the software systems that will operate the driverless race car. 
\end{abstract} 

\thispagestyle{empty}

\section{Problem Definition}
Oregon State University's Global Formula Racing team is consistently one of the best placing teams in the Formula Student competitions. With the addition of a driverless racing category to the GFR competitions, it is necessary for Oregon State University's team to develop a driverless car to maintain their competitive results and fulfill their main team goal of winning competitions. There are three main problems to be solved to implement a fully autonomous vehicle: seeing and interpreting the environment, creating a short term plan for navigation, and executing that plan. These three problems are respectively referred to as \textit{seeing}, \textit{thinking}, and \textit{acting}.

In the \textit{seeing} phase, the vehicle must identify its environment, the track. For two of the events, acceleration and skid platform, the track is known and so does not have to be dynamically built through sensors. For the other two events, the race car must be able to gather outside data and process it in such a way that a virtual representation of the track can be modeled over time. The vehicle must also be able to locate its state within the environment, which will include its position, ground speed, heading, angular velocity, and more.

After the environment has been identified, the \textit{thinking} stage occurs, where onboard computer must plan a path through the modeled track. Careful attention must be paid to be sure that the path chosen keeps the race car fully within the bounds of the track. For optimal planning, the physical capabilities and limitations of the vehicle must be taken into consideration such that the path can be realistically followed. 

When a path through the track is developed, the vehicle begins \textit{acting}. Low-level controllers command mechanical actuators to best follow the planned path. Precision is vitally important, as the path planner expects to be able to continue the path from the planned end state of the previous segment. 

\section{Proposed Solution}
Each solution must be designed with simplicity and reliability in mind. Because the three top level problems detailed earlier are mostly independent with well defined interfaces, each solution can be designed, simulated, and verified independently given some global information. The most important global invariant to each solution is the physical model of the vehicle, which will be used to predict the vehicle's future state and analyze the dynamic properties of the racecar under operation.

The \textit{seeing} phase will require the most software development. First, computer vision must be used to pick out the track from the rest of the computer image. Next, that image will be combined with a lidar image to determine the track's place in space. For the vehicle's internal state, the physical model will be used along with robust statistical analysis to ensure that incoming sensor data is correct and effectively synchronized. Because internal sensors operate at different rates, it will be necessary to process the kinematics of the vehicle to find expected sensor values in intermediate time steps. Both the internal state and the scanned environment will then be used to simultaneously locate the vehicle within the environment and map the environment around it, a process called SLAM.

\textit{thinking} requires some software development to dynamically find an optimized path from a set of possible paths. Computationally, this is a simple optimization problem, which has been solved many times before in the field of computer science. It will make use of the environment model and full vehicle state from the \textit{seeing} phase to first determine possible paths through the track and then evaluate those paths for dynamic feasibility and speed.

The \textit{acting} stage requires the least involvement from the software team, as the extent of the main computer's involvement is resolving the path generated by the \textit{thinking} step into motion profiles for each individual actuator through heavy use of the physical model. The low-level control systems will be implemented on specialized microprocessors decicated to each actuator with access to only the sensor data necessary to adequately control the individual actuator to match the higher level commands.

\section{Solution Metrics}
Oregon State University's Global Formula Racing team's design philosophy is "simplicity and reliability through simulation validated by physical testing". As such, the effectiveness of a solution will be determined by its functionality, simplicity, and reliability.

The most basic metric is the product as a fully functional prototype. The race car must be able to complete a lap around the track. This goal is in place as the most basic of baselines, and indicates that a demonstrable product has been developed for the engineering expo. This also shows the success of the project for the Global Formula Racing team as a development endeavor.

The competitive viability of the product is the next priority goal. Because the track is unknown, it isn't possible to set out an absolute time goal, so the speed requirement of the vehicle will have to be measured comparatively. When operating autonomously, the vehicle must be able to complete a single lap around any given track within 25\% of the time it takes for an experienced human driver, and must be able to complete subsequent laps around the same track within 10\% of the time of an experienced human driver.

Unlike speed, robustness is absolutely measurable. The vehicle must be able to operate at full efficiency while any simple sensor is malfunctioning, and must still be able to complete the track if any core sensor (lidar, camera) is malfunctioning. The vehicle must also be able to tolerate outside disturbances, such as bumps in the track or a slipping wheel, fully correcting for them within 1 second at most.

\nocite{*}
\bibliography{empty}

\end{document}
